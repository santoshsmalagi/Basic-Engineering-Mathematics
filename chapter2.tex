\section{Quadratic Equations}

The quadratic equation 
\begin{equation}
\label{quad}
ax^2+bx+c = 0
\end{equation}

where \(a,b\) and \(c\) are constants and \(a \neq 0\), has two solutions for the variable \(x\):

\begin{equation}
\label{root}
x_{1,2} = \frac{-b \pm \sqrt{b^2 - 4ac}}{2a}
\end{equation}

If the \emph{discriminant} \( \Delta \) with 

\begin{equation}
\label{delta}
\Delta = b^2 - 4ac = 0
\end{equation}

then the equation (\ref{quad}) has real and equal roots given by:

\begin{equation}
\label{equal_root}
x = \frac{-b}{2a}
\end{equation}

\textbf{Sum of the roots}:
\begin{equation}
\label{sum_root}
x1 + x2 = \frac{-b}{a}
\end{equation}

\textbf{Product of the roots}:
\begin{equation}
\label{product_root}
x1.x2 = \frac{c}{a}
\end{equation}