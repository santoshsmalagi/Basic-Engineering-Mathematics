\section{Limits and Differentiation}
\subsection{Fundamental principles of limits}
\begin{enumerate}
\item The limit may exist at a point even if the function is undefined.
\item If a function $f(x)$ is defined at a point $a$ i.e. $f(a)$ exists, it is not necessary that the limit at $a$ must exist. Moreover, even if the limit exists it need not be equal to $f(a)$.
\item \textbf{Indeterminate forms:} Any function assuming either of these forms is said to be indeterminate - $\frac{0}{0}, 0.\infty,\frac{\infty}{\infty},\infty - \infty, 0^0, 1^\infty, \infty^0$.
\end{enumerate}

\vspace{-5mm}

\subsection{Properties}
\vspace{-1mm}
\begin{align*}
&\lim_{x \to a} [f(x) \pm g(x)] = \lim_{x \to a} f(x) \pm \lim_{x \to a} g(x).\\
&\lim_{x \to a} [f(x).g(x)] = \lim_{x \to a} f(x).\lim_{x \to a} g(x).\\
&\lim_{x \to a} \frac{f(x)}{g(x)} = \frac{\lim_{x \to a} f(x)}{\lim_{x \to a} g(x)} \quad \text{provided} \, \lim_{x \to a} g(x) \neq 0.
\end{align*}

\vspace{-5mm}

\subsection{Standard Limits}
\vspace{-1mm}
\begin{align}
&\lim_{x \to 0} \sin x = 0 \\
&\lim_{x \to 0} \tan x = 0 \\
&\lim_{x \to 0} \cos x = 1 \\
&\lim_{x \to 0} \frac{\sin x}{x} = 1 \\
&\lim_{x \to 0} \frac{\tan x}{x} = 1 \\
&\lim_{x \to 0} \frac{\sin^{-1} x}{x} = 1 \\
&\lim_{x \to 0} \frac{\tan^{-1} x}{x} = 1 \\
&\lim_{x \to 0} \frac{a^x - 1}{x} = \ln a \quad \text{where} \hspace{2mm} a>0\\
&\lim_{x \to 0} \frac{e^x - 1}{x} = 1 \\
&\lim_{x \to 0} \frac{\log_a (1+x)}{x} = \log_a e \quad \text{where} \hspace{2mm} a>0 \hspace{2mm} \text{and} \hspace{2mm} a\neq0
\end{align}

\vspace{-2mm}

\begin{align}
&\lim_{x \to 0} \frac{\ln (1+x)}{x} = 1 \\
&\lim_{x \to 0} \frac{\log x}{x^m} = 0 \quad \text{where} \hspace{2mm} m>0 \\
&\lim_{x \to 0} \frac{(1+x)^m - 1}{x} = m \\
&\lim_{x \to 0} \frac{x^n - a^n}{x-a} = n.a^{n-1}
\end{align}

\subsection{Dealing with indeterminate forms}
If $f(x)$ and $g(x)$ are two functions such that $f(x)\rightarrow 0$ as $x \rightarrow a$ and $g(x) \rightarrow \infty$ as $x \rightarrow a$ such that $[1+f(x)]^{g(x)}$ assumes the form $1^\infty$ then:

\begin{tcolorbox}
\begin{center}
\begin{align*}
\lim_{x \to a}{[1+f(x)]}^{g(x)} = {e} ^ {\lim_{x \to a} f(x) . g(x)} 
\end{align*}
\end{center}
\end{tcolorbox} 

\subsection{L'Hospital's Rule}
L'Hospital's rule states that for functions $f(x)$ and $g(x)$ which are differentiable on an open interval $I$ except possibly at a point $c$ contained in $I$, if:

\vspace{5mm}

\noindent
$ \lim_{x \to c} f(x) = \lim_{x \to c} g(x) = 0$ or $\pm \infty $ and $g^{'}(x)\neq 0$ for all $x$ in $I$ with $x \neq c$ and \\

\noindent
$\lim_{x \to a} \frac{f^{'}(x)}{g^{'}(x)}$ exists then:

\begin{tcolorbox}
\begin{center}
\begin{align*}
\lim_{x \to a} \frac{f(x)}{g(x)} = \lim_{x \to a} \frac{f^{'}(x)}{g^{'}(x)}
\end{align*}
\end{center}
\end{tcolorbox} 

\vspace{-3mm}

\subsection{Differentiation Formulae}
\begin{align*}
&\frac{d}{dx}[f(x) \pm g(x)] = \frac{df(x)}{dx} \pm\frac{dg(x)}{dx}\\[5pt]
&\frac{d}{dx}f(x).g(x) = \frac{d}{dx}f(x).\frac{d}{dx}g(x)\\[5pt]
&\frac{dc}{dx} = 0 \quad \text{where c is a real constant} \\[5pt]
&\frac{d x^n}{dx} = nx^{n-1} \quad \text{where n is any real number}\\[5pt]
\vspace{5mm}
&\frac{de^x}{dx} = e^x \\[5pt]
&\frac{da^x}{dx} = a^x\log_{e}a
\end{align*}

\subsection{Differentiation of Trigonometric Functions}
\begin{align*}
&\frac{d}{dx}\sin x = \cos x \\[5pt]
&\frac{d}{dx} \cos x = -\sin x \\[5pt]
&\frac{d}{dx} \tan x = \sec^{2} x \\[5pt]
&\frac{d}{dx} \mathrm{cosec} x = -\mathrm{cosec}x . \cot x \\[5pt]
&\frac{d}{dx}\sec x = \sec x. \tan x \\[5pt]
&\frac{d}{dx}\cot x = - \mathrm{cosec} ^2 x 
\end{align*}

\subsection{Product Rule}
\begin{tcolorbox}
\begin{center}
\begin{align*}
\frac{d}{dx}.uv= u.\frac{dv}{dx} + v.\frac{du}{dx}
\end{align*}
\end{center}
\end{tcolorbox}

\subsection{Quotient Rule}
\begin{tcolorbox}
\begin{center}
\begin{align*}
\frac{d}{dx}.\frac{u}{v} = \frac{v.\frac{du}{dx} - u.\frac{dv}{dx}}{v^2}
\end{align*}
\end{center}
\end{tcolorbox}

\subsection{Derivative of Inverse of a function}
\begin{tcolorbox}
\begin{center}
\begin{align*}
\frac{dy}{dx} = \frac{1}{\frac{dx}{dy}}
\end{align*}
\end{center}
\end{tcolorbox}

\subsection{Chain Rule of Differentiation (Derivative of composite function)}
If $u = \phi(x)$ \\
$y = \phi(u)$, then \\
\begin{tcolorbox}
\begin{center}
\begin{align*}
\frac{dy}{dx} = \frac{dy}{du}.\frac{du}{dx} 
\end{align*}
\end{center}
\end{tcolorbox}