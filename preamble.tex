%------------------------------------------------------------
% Universal Template : Compiled by Santosh S Malagi

% Instructions:

% # Uncomment those packages which are required.
% # Comment packages which are not required.
% # [] - signifies optional arguments
% # {} - compulsory arguments

%----------------------------------------------------------------------------------------
%	LAYOUT AND FORMATTING
%----------------------------------------------------------------------------------------

% Page margins and layout

%\usepackage[a4paper,top=3cm,bottom=3cm,left=1.5cm,right=3cm,bindingoffset=1cm]{geometry}  

\usepackage[a4paper,top=2cm,bottom=3cm,inner=1.5cm,outer=3cm,bindingoffset=1cm]{geometry} 

% English language/hyphenation
\usepackage[english]{babel} 

% Required for specifying colors by name
\usepackage{xcolor} 

%for inserting fancy headers and footers
\usepackage{fancyhdr} 

% Required for formatting table of contents
\usepackage{titletoc} 

% Required for formatting section headings
\usepackage{titlesec}

% Required for formatting list of tables and figures
\usepackage{tocloft}

% Caption and sub-caption properties
%\usepackage{caption}
%\usepackage{subcaption}

% Packages for controlling line spacing
\usepackage{setspace}

%Package for author details
\usepackage{authblk}

%----------------------------------------------------------------------------------------
%	FONTS
%----------------------------------------------------------------------------------------

% Use the Avantgrade font for headings
%\usepackage{avant} 

% Use the Times font for headings
%\usepackage{times} 

% Use the Adobe Times Roman as the default text font together with math symbols from the Symbol, Chancery and Computer Modern fonts
%\usepackage{mathptmx} 

% Slightly tweak font spacing for aesthetics
\usepackage{microtype} 

% Required for including letters with accents
%\usepackage[utf8]{inputenc} 
\usepackage[utf8x]{inputenc}

% Use 8-bit encoding that has 256 glyph
\usepackage[T1]{fontenc} 

%Font pack
\usepackage{lmodern}
 
 
%----------------------------------------------------------------------------------------
%	GRAPHICS AND TABLES
%----------------------------------------------------------------------------------------

% Required for enhanced graphics support
\usepackage[pdftex]{graphicx} 

% Specifies the directory where pictures are stored
\graphicspath{{images/}} 

% Required for drawing custom shapes
\usepackage{tikz} 

%For tabular options
\usepackage{array}

%Automatically adjust table width
%\usepackage{tabularx}

%Merging multiple rows
\usepackage{multirow}

%Controlling spacing within tables
\usepackage{booktabs}

%Formatting table properties
%\usepackage{threeparttable}

%Formatting table style
%\usepackage{longtable}

%To change table orientation to landscape mode
%\usepackage{lscape}

%Increase row height in LaTeX table
\setlength\extrarowheight{25pt}

%Baseline width modification 
%\setlength\baselineskip{5pt}

%----------------------------------------------------------------------------------------
%	HYPERLINKS IN THE DOCUMENTS
%----------------------------------------------------------------------------------------
\usepackage{hyperref}
\usepackage{cleveref}
%\usepackage{url}

%----------------------------------------------------------------------------------------
%	MATHEMATICS
%----------------------------------------------------------------------------------------
\usepackage{amsmath}
\usepackage{amssymb}
\usepackage{amsthm}
\usepackage{amsfonts}
%----------------------------------------------------------------------------------------
%	BIBLIOGRAPHY SUPPORT
%----------------------------------------------------------------------------------------
%Use either of the two

%\usepackage{natbib}
%\usepackage{biblatex}

%----------------------------------------------------------------------------------------
%	MISCELLANEOUS
%----------------------------------------------------------------------------------------
%for SI unit support
%\usepackage{siunitx} 

% package for producing filler text
%\usepackage{blindtext} 

%\usepackage{verbatim} % for comments

%----------------------------------------------------------------------------------------
%	NOTES AND DOCUMENTATION
%----------------------------------------------------------------------------------------
%Package for notes
%\usepackage{todonotes}

%Package for footnote style
%\usepackage{footmisc}

%Package for glossaries
%\usepackage{glossaries}

