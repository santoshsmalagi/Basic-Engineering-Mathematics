\section{Indefinite Integrals}

Evaluation of indefinite integrals can be accomplished using one of the possible methods: \\[2mm]

\begin{itemize}
\item Decomposition of given integral into sum of integrals which can be reduced using standard formulas.

\item Integration by substitution

\item Integration by parts

\item Integration by successive reduction

\end{itemize}

\subsection{Integration by substitution}
If $f(x)$ is a function such that $x$ can be substituted as $ x = \phi(t) $, then: \\[2mm]

\begin{tcolorbox}
\begin{center}
$ \displaystyle \int f(x).dx = \int f(\phi(t)).\phi'(t).dt \quad \text{where} \: x = \phi(t) $
\end{center}
\end{tcolorbox}

\vspace{2mm}

\textbf{Proof:}
\noindent
\begin{align*}
&v = \int f(x).dx = \int f(\phi(t)) \: \text{substituting} \: x = \phi(t) \\
&\frac{dv}{dx} = f(x) \\
&\frac{dv}{dt} = \frac{dv}{dx}.\frac{dx}{dt} \\
&\Rightarrow \frac{dv}{dt} = \frac{dv}{dx}.\frac{dx}{dt} = f(x).\frac{dx}{dt} \\
&\text{Integrating both sides w.r.t we get} \: t \\
&v = \int f(x).\frac{dx}{dt} = \int f(\phi(t)).\phi'(t).dt for  \: x = \phi(t) \\
\end{align*}

\subsubsection{$ \mathbf{\int \frac{f'(x)}{f(x)}.dx = \log{f(x)}} $}

\noindent
\textbf{Proof:}
\begin{align*}
&\text{Let} \: f(x) = t \\
&f'(x) = \frac{dt}{dx} \\
&f'(x).dx = dt \\
&\therefore \int \frac{f'(x)}{f(x)}dx = \int \frac{dt}{t} = \log t = \log f(x) \\
\end{align*}

\noindent
\textit{Integral of a fraction whose numerator is the derivative of the denominator is equal to the logarithm of the denominator.}


\subsubsection{$ \mathbf{\int [f(x)]^n f'(x).dx = \frac{[f(x)]^{n+1}}{n+1}}$ , such that n $\neq$ 0}

\noindent
\textbf{Proof:}
\begin{align*}
&\text{Let} \: f(x) = t \\
&f'(x)dx = dt \\
&\therefore \int [f(x)]^n.f'(x)dx = \int t^n.dt = \frac{t^{n+1}}{n+1} = \frac{[f(x)]^{n+1}}{n+1} \\ 
&\text{for n} \neq -1
\end{align*}

\subsubsection{$ \mathbf{\int f'(ax+b)dx = \frac{f(ax+b)}{a}}$}
\noindent
\textbf{Proof:}
\begin{align*}
&\text{Let} \: ax+b = t \\
&adx = dt \implies \\
&\therefore \int [f(x)]^n.f'(x)dx = \int t^n.dt = \frac{t^{n+1}}{n+1} = \frac{[f(x)]^{n+1}}{n+1} \\ 
&\text{for n} \neq -1
\end{align*}

\noindent
\textit{Integral of a function f(ax+b) is of the same form as that of f(x), divided by the coefficient of x.}

\vspace{5mm}
\noindent
\textbf{Summary}

\vspace{2mm}

\begin{tcolorbox}
\begin{center}
$ \int \frac{f'(x)}{f(x)}.dx = \log{f(x)} $ \\
$ \int [f(x)]^n f'(x).dx = \frac{[f(x)]^{n+1}}{n+1} $ \\
$ \int f'(ax+b)dx = \frac{1}{a}.{f(ax+b)} $
\end{center}
\end{tcolorbox}

\subsection{Integration of Trigonometric Functions}
\begin{align*}
&\int \sin x.dx = -\cos x + c \\[2mm]
&\int \cos x.dx = \sin x + c \\[2mm]
&\int \tan x.dx = -\log |\cos x| + c = \log |\sec x| + c \\[2mm]
&\int \cot x.dx =  \log |\sin x| + c =  -\log |\mathrm{cosec} x| + c \\[2mm]
&\int \sec x.dx = \log |\sec x + \tan x| + c = \log \tan \left[\frac{\pi}{4} + \frac{x}{2} \right] \\[2mm]
&\int \mathrm{cosec} x.dx = \log |\mathrm{cosec}x - \cot x| + c = \log \tan \frac{x}{2}
\end{align*}

\subsection{Standard Integrals}
\begin{align*}
&\int \frac{1}{a^2+x^2}.dx = \frac{1}{a}.\tan^{-1} {\frac{x}{a}} + c = -\frac{1}{a}.\cot^{-1} {\frac{x}{a}} + c \\[3mm]
&\int \frac{1}{x^2-a^2}.dx = \frac{1}{2a}.\log\left|\frac{x-a}{x+a}\right|+c\\[3mm]
&\int \frac{1}{a^2-x^2}.dx = \frac{1}{2a}.\log\left|\frac{a+x}{a-x}\right|+
c \\[6mm]
&\int \frac{1}{\sqrt{a^2-x^2}}.dx = \sin^{-1}\frac{x}{a}+c = -\cos^{-1}\frac{x}{a}\\[3mm]
&\int \frac{1}{\sqrt{a^2+x^2}}.dx = \sinh^{-1}\frac{x}{a}+c = \log \frac{x+\sqrt{x^2+a^2}}{a}\\[3mm]
&\int \frac{1}{\sqrt{x^2-a^2}}.dx = \cosh^{-1}\frac{x}{a}+c = \log \frac{x+\sqrt{x^2-a^2}}{a}\\[6mm]
&\int\sqrt{a^2-x^2}.dx = \frac{1}{2}x.\sqrt{a^2-x^2}+\frac{a^2}{2}\sin^{-1}\frac{x}{a}+c \\[3mm]
&\int\sqrt{a^2+x^2}.dx = \frac{1}{2}x.\sqrt{a^2+x^2}+\frac{a^2}{2}\log\frac{x+\sqrt{x^2+a^2}}{a}+c \\[3mm]
&\int\sqrt{x^2-a^2}.dx = \frac{1}{2}x.\sqrt{x^2-a^2}-\frac{1}{2}.a^2.\log\frac{x+\sqrt{x^2-a^2}}{a} \\[6mm]
&\textbf{Corollary:} \\[2mm]
&\sinh^{-1}\frac{x}{a} = \log\frac{x+\sqrt{x^2+a^2}}{a} \\[3mm]
&\cosh^{-1}\frac{x}{a} = \log\frac{x+\sqrt{x^2-a^2}}{a}
\end{align*}

\subsection{Integration by Parts}
If $u$ and $v$ are two functions of x then: \\
\begin{tcolorbox}
\begin{center}
$ \int u.v.dx = u.\int v.dx - \int\frac{du}{dx}.\left[\int v.dx \right].dx $
\end{center}
\end{tcolorbox}

\vspace{2mm}

\noindent
\textbf{\textit{If no second function is available unity is taken as the second function.}}

\begin{tcolorbox}
\begin{center}
\textbf{\textit{Integral of Product of two functions = First Function x Integral of Second Function - Integral of [Differential of First function x Integral of Second Function] }}
\end{center}
\end{tcolorbox}

\vspace{5mm}

\noindent
\textit{Criteria for choosing first and second function : \textbf{ILATE}}
\begin{itemize}
\item \textit{I:Inverse Trigonometric Function}
\item \textit{L:Logarithmic Function}
\item \textit{A:Algebraic Function}
\item \textit{T:Trigonometric Function}
\item \textit{E:Exponential Function}
\end{itemize}

\vspace{-2mm}

\subsection{Bernoulli's Theorem}

$ \displaystyle \int u.v.dx \rightarrow u $ is an algebraic function which becomes zero after differentiating for finite steps then: 

\vspace{2mm}

\begin{tcolorbox}
\begin{center}
$ \int u.v.dx = u \int v.dx - u'\int\int v.dx + u''\int\int\int v.dx.............$
\end{center}
\end{tcolorbox}

\vspace{2mm}
\noindent
Example: \\
\begin{align*}
&\int {x^2}\cos x.dx = x^2 \int cos x - 2.x \int \int \cos x + 2 \int \int \int cos x \\
& x^2.sin x + 2x \cos x - 2 \sin x
\end{align*}

\subsection{Evaluation of Integrals using $ e^x $}
\subsubsection{$ \mathbf{\int e^x[f(x) + f'(x)].dx} $}
Integrating by parts, \\
\begin{align*}
&\int e^xf(x).dx = e^xf(x) - \int e^x f'(x)dx \\
&\text{Now} \int e^x[f(x)+f'(x)].dx = \int e^xf(x).dx + \int e^xf'(x)dx \\
&\text{We get} 
\end{align*}
\vspace{1mm}
\begin{tcolorbox}
\begin{center}
$ \mathbf{\int e^x[ f(x) + f'(x) ].dx = e^x.f(x)} $
\end{center}
\end{tcolorbox}
\subsubsection{$ \mathbf{\int e^{ax}\cos(bx+c)} $}
\begin{tcolorbox}
\begin{center}
$ \int e^{ax} \cos (bx+c).dx = \frac{e^{ax}}{a^2+b^2}.(a \cos (bx+c)+ b \sin (bx+c))
= \frac{e^{ax} \cos \left( bx+c - \tan^{-1}\frac{b}{a} \right)}{\sqrt{a^2+b^2}} $
\end{center}
\end{tcolorbox}
\subsubsection{$ \mathbf{\int e^{ax}\sin(bx+c)} $}
\begin{tcolorbox}
\begin{center}
$ \int e^{ax} \sin (bx+c).dx = \frac{e^{ax}}{a^2+b^2}.(a \sin (bx+c) - b \cos (bx+c))
= \frac{e^{ax} \sin \left( bx+c - \tan^{-1}\frac{b}{a} \right)}{\sqrt{a^2+b^2}} $
\end{center}
\end{tcolorbox}

\subsection{Integration of Rational Algebraic Expressions}
\textbf{Rational Function:} An expression of the form $\frac{f(x)}{g(x)}$ where $f(x)$ and $g(x)$ are polynomial functions given by:

\vspace{-3mm}

\[ f(x) = a_0x^m + a_1x^{m-1}..........a_{m-1}x + a_m \] 

\vspace{-5mm}

\[ g(x) = b_0x^n + b_1x^{n-1}..........n_{n-1}x + b_n \]

\noindent
\textbf{Proper Rational Function:} Degree of Numerator $<$ Denominator.

\vspace{2mm}

\noindent
\textbf{Improper Rational Function:} Degree of Numerator $ \geq $ degree of Denominator.

\vspace{3mm}

\noindent
\textbf{\textit{Expressions involving proper rational functions can be integrated by resolving them into partial fractions.}}

\vspace{5mm}

\noindent
\begin{center}
\begin{tabular}{|c|c|c|}
\hline 
S.No & Type of Factors in the denominator & Form of the partial fraction \\ 
\hline 
1. & Linear and non repeated of the form $ (ax+b) $ & $ \frac{A}{ax+b} $ \\ 
\hline 
2. & Linear repeated factors, $ (ax+b)^p $ & $\frac{A}{(ax+b)} + \frac{B}{(ax+b)^2}+.........\frac{N}{(ax+b)^p} $ \\ 
\hline 
3. & Quadratic non repeated factors $ (ax^2 + bx +c ) $ & $\frac{Ax+b}{ax^2+bx+c} $ \\ 
\hline 
4. & Quadratic repeated factor $ (ax^2+bx+c)^p $ & $ \frac{A_1x+B_1}{ax^2+bx+c} + \frac{A_2x+B_2}{ax^2+bx+c}+...... $ \\
\hline 
\end{tabular} 
\end{center}


\vspace{5mm}
\begin{center}
\textbf{***********}
\end{center}