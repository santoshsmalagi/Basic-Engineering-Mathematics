\section{Indefinite Integrals}

Evaluation of indefinite integrals can be accomplished using one of the possible methods: \\[2mm]

\begin{itemize}
\item Decomposition of given integral into sum of integrals which can be reduced using standard formulas.

\item Integration by substitution

\item Integration by parts

\item Integration by successive reduction

\end{itemize}

\subsection{Integration by substitution}
If $f(x)$ is a function such that $x$ can be substituted as $ x = \phi(t) $, then: \\[2mm]

\begin{tcolorbox}
\begin{center}
$ \displaystyle \int f(x).dx = \int f(\phi(t)).\phi'(t).dt \quad \text{where} \: x = \phi(t) $
\end{center}
\end{tcolorbox}

\vspace{2mm}

\textbf{Proof:}
\noindent
\begin{align*}
&v = \int f(x).dx = \int f(\phi(t)) \: \text{substituting} \: x = \phi(t) \\
&\frac{dv}{dx} = f(x) \\
&\frac{dv}{dt} = \frac{dv}{dx}.\frac{dx}{dt} \\
&\Rightarrow \frac{dv}{dt} = \frac{dv}{dx}.\frac{dx}{dt} = f(x).\frac{dx}{dt} \\
&\text{Integrating both sides w.r.t we get} \: t \\
&v = \int f(x).\frac{dx}{dt} = \int f(\phi(t)).\phi'(t).dt for  \: x = \phi(t) \\
\end{align*}

\subsubsection{$ \mathbf{\int \frac{f'(x)}{f(x)}.dx = \log{f(x)}} $}

\noindent
\textbf{Proof:}
\begin{align*}
&\text{Let} \: f(x) = t \\
&f'(x) = \frac{dt}{dx} \\
&f'(x).dx = dt \\
&\therefore \int \frac{f'(x)}{f(x)}dx = \int \frac{dt}{t} = \log t = \log f(x) \\
\end{align*}

\noindent
\textit{Integral of a fraction whose numerator is the derivative of the denominator is equal to the logarithm of the denominator.}


\subsubsection{$ \mathbf{\int [f(x)]^n f'(x).dx = \frac{[f(x)]^{n+1}}{n+1}}$ , such that n $\neq$ 0}

\noindent
\textbf{Proof:}
\begin{align*}
&\text{Let} \: f(x) = t \\
&f'(x)dx = dt \\
&\therefore \int [f(x)]^n.f'(x)dx = \int t^n.dt = \frac{t^{n+1}}{n+1} = \frac{[f(x)]^{n+1}}{n+1} \\ 
&\text{for n} \neq -1
\end{align*}

\subsubsection{$ \mathbf{\int f'(ax+b)dx = \frac{f(ax+b)}{a}}$}
\noindent
\textbf{Proof:}
\begin{align*}
&\text{Let} \: ax+b = t \\
&adx = dt \implies \\
&\therefore \int [f(x)]^n.f'(x)dx = \int t^n.dt = \frac{t^{n+1}}{n+1} = \frac{[f(x)]^{n+1}}{n+1} \\ 
&\text{for n} \neq -1
\end{align*}

\noindent
\textit{Integral of a function f(ax+b) is of the same form as that of f(x), divided by the coefficient of x.}

\vspace{5mm}
\noindent
\textbf{Summary}

\vspace{2mm}

\begin{tcolorbox}
\begin{center}
$ \int \frac{f'(x)}{f(x)}.dx = \log{f(x)} $ \\
$ \int [f(x)]^n f'(x).dx = \frac{[f(x)]^{n+1}}{n+1} $ \\
$ \int f'(ax+b)dx = \frac{1}{a}.{f(ax+b)} $
\end{center}
\end{tcolorbox}

\subsection{Integration of Trigonometric Functions}
\begin{align*}
&\int \sin x.dx = -\cos x + c \\[2mm]
&\int \cos x.dx = \sin x + c \\[2mm]
&\int \tan x.dx = -\log |\cos x| + c = \log |\sec x| + c \\[2mm]
&\int \cot x.dx =  \log |\sin x| + c =  -\log |\mathrm{cosec} x| + c \\[2mm]
&\int \sec x.dx = \log |\sec x + \tan x| + c = \log \tan \left[\frac{\pi}{4} + \frac{x}{2} \right] \\[2mm]
&\int \mathrm{cosec} x.dx = \log |\mathrm{cosec}x - \cot x| + c = \log \tan \frac{x}{2}
\end{align*}