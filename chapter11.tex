\section{Linear Inequalities}

\setlength{\baselineskip}{1pt}
Inequalities of the form: %\\
\vspace{\baselineskip}
\[ax + b < 0\] %\\
\vspace{\baselineskip}
 \[ax + b > 0\] %\\
\vspace{\baselineskip} 
\[ax + b \leq 0\] %\\
\vspace{\baselineskip} 
\[ax + b \geq 0\] %\\
\vspace{\baselineskip} 
are known as linear inequalities.
\vspace{\baselineskip}
 
\begin{enumerate}
\item Add and subtract any number $k$ without change in the inequality on both sides.
\item When multiplying or dividing by constant $k$, reverse the sign of the inequality only when $k$ is negative.
\item Linear inequalities have infinite solution sets, these can be obtained by isolating the variable and solving them in a manner similar to solving equations.\\ \\
\end{enumerate}

\begin{center}
\begin{tabular}{|c|c|>{\centering\arraybackslash}p{0.5\textwidth}|}
\hline 
Inequality & Representation & Graph \\ 
\hline 
$ a < x < b$ & (a,b) &  \\ 
\hline 
$ a \leq x \leq b $ & $ [a,b] $ &  \\ 
\hline 
$ a < x \leq b $ & $ (a,b] $ &  \\ 
\hline 
$ a \leq x < b $ & $ [a,b) $ &  \\ 
\hline 
$ x > a $ & $ (a,\infty) $ &  \\ 
\hline 
$ x \geq a $ & $ [a,\infty) $ &  \\ 
\hline 
$ x < b $ & $ (-\infty,b) $ &  \\ 
\hline 
$ x \leq b $ & $ (-\infty,b] $ &  \\ 
\hline 
\end{tabular} 
\end{center}

\vspace{1mm}
\begin{center}
\textbf{***********}
\end{center}