\section{Exponential Functions}
\begin{equation}
\label{amn}
a^m . a^n  = a^{m+n}
\end{equation}
\begin{equation}
\label{am/n}
\frac{a^m}{a^n}  = a^{m-n}
\end{equation}
\begin{equation}
\label{a^mn}
({a^m})^{n} = a^{mn}
\end{equation}
\begin{equation}
\label{am_equal_an}
a^m = a^n \implies m = n
\end{equation}
\begin{equation}
\label{atominusn}
a^{-n} = \frac{1}{a^n} \hspace{10mm} 
\text{(for a being any non zero real number)}
\end{equation}
\begin{equation}
\label{ato0}
a^{0} = 1
\end{equation}

\subsection{Rational Number Exponents}
The principle nth root of x>0 is defined as: \\
\begin{align*}
x^{1/n} &= 
\begin{cases}
\text{is a unique real number y if n is odd or even} \\
0, if x = 0 \\
\text{not a real number if x<0}\\
\end{cases}
\\
(x^{1/n})^{n} &= 
\begin{cases}
\text{x,if n is odd/even and x is positive} \\
\text{|x|, if n is even and x is negative}\\
\end{cases}
\end{align*}




