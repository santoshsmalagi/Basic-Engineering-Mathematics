\section{Laws of Exponents}
\begin{equation}
\label{xmn}
x^a . x^b  = x^{a+b}
\end{equation}
\begin{equation}
\label{xm/n}
\frac{x^a}{x^b}  = x^{a-b}
\end{equation}
\begin{equation}
\label{x^mn}
({x^a})^{b} = x^{ab}
\end{equation}
\begin{equation}
\label{xm_equal_xn}
x^a = y^b \implies a = b
\end{equation}
\begin{equation}
\label{xtominusn}
x^{-a} = \frac{1}{x^a} \hspace{10mm} 
\text{(for a being any non zero real number)}
\end{equation}
\begin{equation}
\label{xto0}
x^{0} = 1
\end{equation}
\begin{equation}
\label{nrootxy}
\sqrt[n]{\frac{x}{y}} = \frac{\sqrt[n]{x}}{\sqrt[n]{y}}
\end{equation}
\begin{equation}
\label{nxy}
\sqrt[n]{xy} = \sqrt[n]{x}.\sqrt[n]{y}
\end{equation}
\begin{equation}
\label{xy-m}
\left( \frac{x}{y} \right)^{-m} = \left( \frac{y}{x} \right)^{m}
\end{equation}
\begin{equation}
\label{xynm}
\frac{x^{-n}}{y^{-m}} = \frac{y^m}{x^n}
\end{equation}

\subsection{Rational Number Exponents}
The principle nth root of x>0 is defined as: \\
\begin{align*}
x^{1/n} &= 
\begin{cases}
\text{is a unique real number y if n is odd or even} \\
0, if x = 0 \\
\text{not a real number if x<0}\\
\end{cases}
\\
(x^{1/n})^{n} &= 
\begin{cases}
\text{x,if n is odd/even and x is positive} \\
\text{|x|, if n is even and x is negative}\\
\end{cases}
\end{align*}

\subsection{Simplest Radical Form}
No radicand can contain a factor with an exponent greater than or equal to the index of the radical e.g. $ \sqrt[3]{16x^3y^5} \implies $ is not in it's standard form. No power of a radicand and index of the radical can have common factor other than 1. e.g. $ 2xy\sqrt[3]{2y^2} \implies $ not in standard form. No radical appears in denominator. No fractional part appears in radical.
e.g. $ \sqrt[6]{t^3} = \sqrt{t} $ is in it's standard form.