\section{Polynomial Functions}

\noindent
\textbf{Zeros of a polynomial:}{ }If $f(c) = 0$, then $c$ is called a zero of the polynomial $f(x)$. \\

\noindent
\textbf{Division algorithm for a polynomial:}{ }If $f(x)$ and $g(x)$ are two polynomials such that $g(x)\neq0$ then there exist two unique polynomials $q(x)$ and $r(x)$ such that: \\

\noindent
\begin{tcolorbox}
\begin{align}
\frac{f(x)}{g(x)} = q(x) + \frac{r(x)}{g(x)}
\end{align}
\end{tcolorbox}

\noindent
\textbf{Fundamental Theorem of Algebra:}{ }Every polynomial with positive degree has at least one complex zero. \\

\noindent
\textbf{Remainder Theorem:}{ }If a polynomial $f(x)$ is divided by $x-c$ then the remainder is $f(c)$. \\

\noindent
\textbf{Factor Theorem:}{ }Polynomial $f(x)$ has a factor $x-c$ if and if only $f(c) = 0$.

\subsection{Corollary to the Fundamental Theorem}
\begin{enumerate}
\item Every polynomial of positive degree $n$ has a factorization of the form:

\begin{tcolorbox}
\begin{align}
P(x) = a_n (x-r_1)(x-r_2).........(x-r_n)
\end{align}
\end{tcolorbox}

where $r_i$ are not necessarily distinct. If $(x-r_i)$ occurs $n$ times it is said to have a multiplicity of $n$.

\item It is not always possible to find the factors using exact methods.

\item A polynomial of degree $n$ has at most $n$ complex zeros.

\item Complex zeros of real polynomials with real coefficients occur in complex conjugate pairs.

\item Any polynomial of degree $n>0$ with real coefficients has complete factorization using linear and quadratic factors, multiplied by the leading coefficient of the polynomial.

\item \textbf{Intermediate Value Theorem}: Given a polynomial $f(x)$ with $a<b$ and $f(a) \neq f(b)$ then $f(c)$ takes on every value $c$ in the interval $(a,b)$.

\item For a polynomial $f(x)$ if $f(a)$ and $f(b)$ have opposite signs then $f(x)$ has at least one zero between $a$ and $b$.

\item For a polynomial $f(x)$ if $f(a)$ and $f(b)$ have opposite signs then $f(x)$ has at least one zero between $a$ and $b$.

\item \textbf{Descartes rule of signs}: If $f(x)$ is a polynomial with terms arranged in descending order, then the number of positive real roots of $f(x)$ is either equal to the number of sign changes between the successive terms of $f(x)$ or less than this number by an even number. Number of negative real zeros of $f(x)$ is obtained by applying this rule to $f(-x)$.

\end{enumerate}

\subsection{Synthetic Division}

$c$ is said to be a zero of the polynomial $f(x)$ if $f(c) = 0$ $\implies$ $x-c$ is a factor of the polynomial $f(x)$. The graph of $f(x)$ has an intercept at $c$.

\begin{enumerate}
\item Arrange the coefficients in descending order in the first row.

\item Third row is formed by bringing down the first coefficient of $f(x)$ then successively multiplying each coefficient in the third row by $c$, placing the results in second row adding this to the corresponding coefficients in the first row, and placing result in the next position of the third row.
\end{enumerate}

\vspace{5mm}
\begin{center}
\textbf{***********}
\end{center}
 